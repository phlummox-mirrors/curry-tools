\section{\texttt{cpm}: The Curry Package Manager}
\label{sec-cpm}

The Curry package manager (CPM) is a tool to
distribute and install Curry libraries and applications
and manage version dependencies between these libraries.
Since CPM offers a lot of functionality, there is a separate
manual available.\footnote{\url{http://curry-language.org/tools/cpm}}
Therefore, we describe here only some basic CPM commands.

The executable \code{cpm} is located in the \code{bin} directory
of \CYS. Hence, if you have this directory in your path,
you can start CPM by cloning a copy of the central package index repository:
%
\begin{curry}
> cpm update
\end{curry}
%
Now you can show a short list of all packages in this index by
%
\begin{curry}
> cpm list
Name             Synopsis                                             Version   
----             --------                                             -------   
addtypes         A tool to add missing type signatures in a Curry     0.0.1     
                 program                                                        
binint           Libraries with a binary representation of natural    0.0.1     
                 and integers                                                   
$\ldots$
\end{curry}
%
The command
%
\begin{curry}
> cpm info PACKAGE
\end{curry}
%
can be used to show more information about the package with name
\code{PACKAGE}.

Some packages do not contain only useful libraries
but also tools with some binary. In order to install such tools,
one can use the command
%
\begin{curry}
> cpm installapp PACKAGE
\end{curry}
%
This command checks out the package in some internal directory
(\code{\$HOME/.cpm/app_packages})
and installs the binary of the tool provided by the package
in \code{\$HOME/.cpm/bin}.
Hence it is recommended to add this directory to your path.

For instance, the most recent version of CPM
can be installed by the following commands:
%
\begin{curry}
> cpm update
$\ldots$
> cpm installapp cpm
$\ldots$ Package 'cpm-xxx' checked out $\ldots$
$\ldots$
INFO  Installing executable 'cpm' into '/home/joe/.cpm/bin'
\end{curry}
%
Now, the binary \code{cpm} of the most recent CPM version can be used
if \code{\$HOME/.cpm/bin} is in your path
(before \code{\cyshome{}/bin}!).

A detailed description how to write your own packages
with the use of other packages can be found in the manual of CPM.

