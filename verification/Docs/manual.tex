\section{Curry2Verify: A Tool to Support the Verification of Curry Programs}
\label{sec-curry2verify}

Curry2Verify\index{Curry2Verify}\index{verifying programs}\index{program!verification}
is a tool that supports the verification of Curry programs
with the help of other theorem provers or proof assistants.
Basically, Curry2Verify extends CurryCheck (see Section~\ref{sec-currycheck}),
which tests given properties of a program,
by the possibility to verify these properties.
For this purpose, Curry2Verify translates properties
into the input language of other theorem provers or proof assistants.
This is done by collecting all operations directly or indirectly
involved in a given property and translating them together with
the given property.

Currently, only Agda \cite{Norell09} is supported as
a target language for verification (but more target languages
may be supported in future releases).
The basic schemes to translate Curry programs into Agda programs
are presented in \cite{AntoyHanusLibby16}.
That paper also describes the limitations of this approach.
Since Curry is a quite rich programming language,
not all constructs of Curry are currently supported
in the translation process (e.g., no case expressions,
local definitions, list comprehensions, do notations, etc).
Only a kernel language, where the involved rules
correspond to a term rewriting system, are translated into Agda.
However, these limitations might be relaxed in future releases.
Hence, the current tool should be considered as a first prototypical
approach to support the verification of Curry programs.

\subsection{Options}

To translate the properties of a Curry program stored
in the file \code{prog.curry} into Agda,
one can invoke the command\pindex{curry2verify}
%
\begin{curry}
curry2verify prog
\end{curry}
%
This generates for each property $p$ in module \code{prog}
an Agda program \ccode{TO-PROVE-$p$.agda}.
If one completes the proof obligation in this file,
the completed file should be renamed into
\ccode{PROOF-$p$.agda}.
This has the effect that CurryCheck does not test this property again
but trusts the proof and use this knowledge to simplify other tests.

The command \code{curry2verify} can be parameterized
with various options which can also be shown using the option
\ccode{--help} and are described in the following.

\begin{description}
\item{\code{-h}, \code{-?}, \code{--help}}
These options trigger the output of usage information.

\item{\code{-q}, \code{--quiet}}
Run quietly and produce no informative output.
However, the exit code will be non-zero if some translation error occurs.

\item{\code{-v[$n$]}, \code{--verbosity[=$n$]}}
Set the verbosity level to an optional value.
The verbosity level \code{0} is the same as option \code{-q}.
The default verbosity level \code{1} shows the translation progress.
The verbosity level \code{2} (which is the same as omitting the level)
shows also the generated (Agda) program.
The verbosity level \code{3} shows also more details about
the translation process.

\item{\code{-n}, \code{--nostore}}
Do not store the translated program in a file but show it only.

\item{\code{-p $p$}, \code{--property=$p$}}
As a default, all properties occurring in the source program are
translated. If this option is provided, only property $p$ is translated.

\item{\code{-t $t$}, \code{--target=$t$}}
Define the target language of the translation.
Currently, only $t = \code{Agda}$ is supported, which is also the
default.

\item{\code{-s $s$}, \code{--scheme=$s$}}
Define the translation scheme used to represent Curry programs
in the target language.

For the target \code{Agda}, the following schemes are supported:
\begin{description}
\item[\code{choice}]
Use the ``planned choice'' scheme, see \cite{AntoyHanusLibby16}
(this is the default).
In this scheme, the choices made in a non-deterministic computation
are abstracted by passing a parameter for these choices.
\item[\code{nondet}]
Use the ``set of values'' scheme, see \cite{AntoyHanusLibby16},
where non-deterministic values are represented in a tree structure.
\end{description}

\end{description}

% LocalWords: CurryCheck
