\section{ERD2Curry: A Tool to Generate Programs from ER Specifications}
\label{sec-erd2curry}

ERD2Curry\index{ERD2Curry}\index{database programming}
is a tool to generate Curry code to access and manipulate data
persistently stored from
entity relationship diagrams.\index{entity relationship diagrams}
The idea of this tool is described in detail in
\cite{BrasselHanusMueller08PADL}.
Thus, we describe only the basic steps to use this tool
in the following.

If one creates an entity relationship diagram (ERD)
with the Umbrello UML Modeller, one has to store its
XML description in XMI format (as offered by Umbrello)
in a file, e.g., \ccode{myerd.xmi}.
This description can be compiled into a Curry program by the
command\pindex{erd2curry}
\begin{curry}
erd2curry -x myerd.xmi
\end{curry}
(\code{erd2curry} is a program stored in \code{\cyshome/bin}
where \cyshome is the installation directory of \CYS;
see Section~\ref{sec-general}).
If \code{MyData} is the name of the ERD, the Curry program file
\ccode{MyData.curry} is generated containing all the necessary
database access code as described in \cite{BrasselHanusMueller08PADL}.
In addition to the generated Curry program file,
two auxiliary program files
\code{ERDGeneric.curry} and \code{KeyDatabase.curry}
are created in the same directory.

If one does not want to use the Umbrello UML Modeller,
which might be the preferred method since the interface to the
Umbrello UML Modeller is no longer actively supported,
one can also create a textual description of the ERD
as a Curry term of type \code{ERD}
(w.r.t.\ the type definition given in module
\code{\cyshome/currytools/erd2curry/ERD.curry})
and store it in some file, e.g., \ccode{myerd.term}.
This description can be compiled into a Curry program by the
command\pindex{erd2curry}
\begin{curry}
erd2curry -t myerd.term
\end{curry}
%
The directory \code{\cyshome/currytools/erd2curry/}
contains two examples for such ERD term files:
\begin{description}
\item[\code{Blog.erdterm}:]
This is a simple ERD model for a blog with entries, comments,
and tags.
\item[\code{Uni.erdterm}:]
This is an ERD model for university lectures as
presented in the paper \cite{BrasselHanusMueller08PADL}.
\end{description}
%
There is also the possibility to visualize an ERD term
as a graph with the graph visualization program \code{dotty}
(for this purpose, it might be necessary to adapt the definition
of \code{dotviewcommand} in your \ccode{\curryrc} file,
see Section~\ref{sec-customization},
according to your local environment).
The visualization can be performed by the command
\begin{curry}
erd2curry -v myerd.term
\end{curry}

